\documentclass{article}
\usepackage[utf8]{inputenc}
\usepackage{calc}
\usepackage{tikz}
\usepackage{array}
\usepackage{polynom}

\title{Adv. Functions Cheat Sheet}
\author{Alex Lavallee }
\date{December 2022}

\begin{document}

\include{longdiv}

\maketitle

\section{Polynomial Functions}
Function Notation: $f(x)=d(x)q(x)+r(x)$, which is equivalent to the division statement of any polynomial, and where: \begin{itemize}
    \item $f(x)$ is the dividend
    \item $d(x)$ is the divisor
    \item $q(x)$ is the quotient
    \item $r(x)$ is the remainder
\end{itemize}

If the remainder is $0$, then the divisor and quotient are both factors of the dividend. The two methods of dividing polynomials are displayed below. They are both the division of $(3x^3-5x^2+8x-1) \div (x-3)$. \\

Polynomial Long Division: 
\polylongdiv{3x^3-5x^2+8x-1}{x-3} \\
\\Synthetic Division:
\polyhornerscheme[x=3]{3x^3-5x^2+8x-1} \\
\\When doing synthetic division, which requires the divisor to be of degree $1$, remember to use $0$ as a placeholder where there is an absent term relative to the range of degrees present in the dividend. \\ \\

Remainder Theorem states that if $f(x)$ is divided by $(x-p)$, the remainder is equal to $f(p)$. Alternatively, if the divisor is $(ax-p)$, then the remainder is $f(\frac{p}{a})$. \\ \\\

Factor Theorem states that $(x-p)$ is a factor of $f(x)$ iff $f(p)=0$. Note that you will have to guess and check in order to determine the first factor, and then division may be used from there. \\ \\

Difference of Cubes: $x^3-y^3=(x-y)(x^2+xy+y^2)$ \\ \\
Sum of Cubes: $x^3+y^3=(x+y)(x^2-xy+y^2)$ \\ \\

Given these factoring strategies, when solving polynomials the suggested order of strategies to be used is: \begin{enumerate}
    \item Common Factor
    \item Difference of Squares
    \item Trinomials
    \item Grouping
    \item Sum/Difference of Cubes
    \item Factor Theorem
\end{enumerate}







\end{document}
