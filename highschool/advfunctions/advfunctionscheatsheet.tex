\documentclass{article}
\usepackage[utf8]{inputenc}
\usepackage{calc}
\usepackage{tikz}
\usepackage{array}
\usepackage{polynom}
\usepackage{amsfonts}

\title{Adv. Functions Cheat Sheet}
\author{Alex Lavallee }
\date{December 2022}

\begin{document}

\include{longdiv}

\maketitle

\section{Polynomial Functions}
Function Notation: $f(x)=d(x)q(x)+r(x)$, which is equivalent to the division statement of any polynomial, and where: \begin{itemize}
    \item $f(x)$ is the dividend
    \item $d(x)$ is the divisor
    \item $q(x)$ is the quotient
    \item $r(x)$ is the remainder
\end{itemize}

If the remainder is $0$, then the divisor and quotient are both factors of the dividend. The two methods of dividing polynomials are displayed below. They are both the division of $(3x^3-5x^2+8x-1) \div (x-3)$. \\

Polynomial Long Division: 
\polylongdiv{3x^3-5x^2+8x-1}{x-3} \\
\\Synthetic Division:
\polyhornerscheme[x=3]{3x^3-5x^2+8x-1} \\
\\When doing synthetic division, which requires the divisor to be of degree $1$, remember to use $0$ as a placeholder where there is an absent term relative to the range of degrees present in the dividend. \\ \\

Remainder Theorem states that if $f(x)$ is divided by $(x-p)$, the remainder is equal to $f(p)$. Alternatively, if the divisor is $(ax-p)$, then the remainder is $f(\frac{p}{a})$. \\ \\\

Factor Theorem states that $(x-p)$ is a factor of $f(x)$ iff $f(p)=0$. Note that you will have to guess and check in order to determine the first factor, and then division may be used from there. \\ \\

Difference of Cubes: $x^3-y^3=(x-y)(x^2+xy+y^2)$ \\ \\
Sum of Cubes: $x^3+y^3=(x+y)(x^2-xy+y^2)$ \\ \\

Given these factoring strategies, when solving polynomials the suggested order of strategies to be used is: \begin{enumerate}
    \item Common Factor
    \item Difference of Squares
    \item Trinomials
    \item Grouping
    \item Sum/Difference of Cubes
    \item Factor Theorem
\end{enumerate}

For graphing polynomial functions, A polynomial function of degree $n$ has: 
\begin{itemize}
    \item at most $n$ solutions
    \item at most $n-1$ turns
    \item the domain is $x \in \mathbb{R}$ \\
\end{itemize}

for polynomial functions of an even degree, the following is true: \begin{itemize}
    \item If $a > 0$, the function will go from QI to QII or vice versa.
    \item If $a < 0$, the function will go from QIII to QIV or vice versa.
    \item There may be no real solutions to the function.\\
\end{itemize}
for polynomial functions of an odd degree, the following is true: \begin{itemize}
    \item If $a > 0$, the function will go from QI to QIII or vice versa.
    \item if $a < 0$, the function will go from QII to QIV or vice versa.
    \item There must be at least one real solution to the function.\\
\end{itemize}

Sketching Algorithm: \begin{itemize}
    \item Standard form must be converted to factored form.
    \item The degree indicates the number of terms, as indicated above.
    \item The leading coefficient indicates the end behaviour.
    \item identify x and y intercepts.
    \item Plot the intercepts and carefully connect the curve they form.
\end{itemize}

Families of functions: Given a general description of a polynomial, a family of curves may be established that satisfy the description. To find a specific member of that family, a point on the function must be substituted into the family function.

\section{Rational Functions}

A rational function is a function that satisfies the following: \\ $f(x)=\frac{p(x)}{q(x)}, q(x) \neq 0$

Vertical Asymptotes: An asymptote is a line that a graph approaches. It can be a type of discontinuity, that is to say the function is not continuous at the asymptote. For a vertical asymptote to exist, there must be a restriction on the independent variable, often such that the denominator of the function cannot equal $0$. In order to describe the behaviour of vertical asymptotes, we use limits. For example, given the graph $f(x)=\frac{1}{x+2}$, we know there is a VA at $-2$, as: \\

$\lim_{x\to -2^-} = -\infty$ \\
and; \\
$\lim_{x\to -2^+} = \infty$ \\

It is said that a VA exists when the limit from either side is $\infty$ or $-\infty$. Note that just because there seems to be a restriction, does not mean it is an asymptote. In the case that the function may be factored to seemingly remove the restriction, a discontinuity would be present in its place. \\

As for HAs, it is said that they occur when the degree of the numerator is less than or equal to the degree of the denominator. You may determine the presence of a HA by taking the limit of the function as x approaches positive or negative infinity. The HA will be slanted (oblique) if the degree of the numerator is greater than the degree of the denominator. An effective means for sketching rational functions is to sketch it in the following order: \begin{itemize}
    \item Factor if possible.
    \item Determine VAs.
    \item Determine HAs including Above/Below.
    \item Graph the VAs and HAs.
    \item Determine key points and sketch. \\
\end{itemize}

Rational Inequalities may easily be solved in this course, as the graph of them must be given. From there you must simply build the set which satisfies the question, although remember to indicate that x is an element of the reals. As for solving rational equations, you will often use the conjugates of the expressions to solve it. Remember that any restrictions must not be included in the final solution, and that complex solutions are valid.

\section{Logarithms and Exponents}

Exponent Laws: \begin{enumerate}
    \item $a^0=1$
    \item $a^m \times a^n = a^{m+n}$
    \item $a^m \div a^n = a^{m-n}$
    \item $(a^m)^n=a^{mn}$
    \item $(ab)^n = a^nb^n$
    \item $a^{-n}=\frac{1}{a^n}$
    \item $(\frac{a}{b})^{-n}=(\frac{b}{a})^n$
    \item $a^{\frac{1}{n}}=\sqrt[n]{a}$
    \item $a^{\frac{m}{n}}=(\sqrt[n]{a})^m$
\end{enumerate}

Logarithmic Form: $y=\log_{b}x$, $b > 0, b\neq 1, x > 0$. \\

Properties of Logarithms: \begin{enumerate}
    \item $log_{a}1 = 0$
    \item $log_{a}a=1$
    \item $log_{a}a^n=n$
    \item $a^{log_{a}x}=x$
    \item $log_{a}xy=log_{a}x+log_{a}y$
    \item $log_{a}\frac{x}{y}=log_{a}x-log_{a}y$
    \item $log_{a}x^n=nlog_{a}x$
    \item $log_{a}\sqrt[n]{x}=\frac{1}{n}log_{a}x$
\end{enumerate}

To solve exponential equations, first attempt to express the equation as the same base. If not, then express all terms as a logarithm of the same base. For logarithmic equations, evaluate the logs on both sides to solve, or convert from log form to exponential form. \\

As for graphing logarithmic functions, assuming there are no positional translations, it will always have a solution at $1$, and a VA at $0$. Every other point of interest will be a power of the base. Alterations follow the standard form of graphical shifts/stretches. This extends to exponential growth and decay problems, which have the following form: \\

$A = A_0b^n$, where: \begin{itemize}
    \item $A$ is the final amount
    \item $A_0$ is the initial amount
    \item $b$ is the growth/decay factor
    \item $n$ is the number of growth/decay periods; which extends to $n = \frac{t}{p}$, $t$ = time, $p$ = length of period
\end{itemize}

Note that when $b > 1$, it is exponential growth, and when $0 < b < 1$, it is exponential decay. $b$ is usually calculated as $b = 1 \pm r$. \\

Other questions that apply logarithms are those that involve the logarithmic scale. Here are some of the problems you may encounter: \\

$M = logI$, $10^M=I$, $M$ is magnitude of earthquake, $I$ is intensity of earthquake. \\ \\
$L = 10log\frac{I}{I_0}$, $I_0(10^\frac{L}{10}) = I$, $L$ is loudness in db, $I$ is intensity of measured sound, $I_0$ is intensity of the softest sound. \\ \\
$pH = -log[H^+]$, $[H^+]$ is the hydrogen concentration in $molL^{-1}$.

\section{Trig Identities}

A basic understanding of how to solve trig equations is already assumed. You should be familiar with the CAST rule, identity triangles, and working with the basic trig functions. It may also be helpful to know that for any solution that has no restrictions, the solution can generally be expressed as: \\ 

$\theta \pm pk\pi$, $k\in \mathbb{Z}$, $p=$ periodicity of the solution. \\

Note that this does not hold true for all trig functions, and must be re-evaluated on a case by case basis. \\

Radians are a ratio of the arc length of a circle to the length of its radius. A full rotation of a circle is $2\pi$ radians, and they are unitless, however you must specify that they are radians. The length of a specified arc can be calculated with the formula $l=\theta r$. Trig equations may also be solved with radians as opposed to degrees, and the conversion ratio is $\pi _{rads} = 180^{\circ}$. \\

A co-terminal angle of a given angle is a differing angle that gives the same result when evaluated. These can be determined by the periodicity of the angle or solution. \\

There are various identities that may be used to solve a trig equation. The addition identities are: \\

$sin(A + B) = sinAcosB - sinBcosA$ \\
$sin(A-B) = sinAcosB - sinBcosA$ \\
$cos(A + B) = cosAcosB - sinAsinB$ \\
$cos(A - B) = cosAcosB + sinAsinB$ \\

The double angle identities are: \\

$sin2A = 2sinAcosA$ \\
$cos2A=cos^2A-sin^2A$ \\
$cos2A = 2cos^2A - 1$ \\
$cos2A = 1 - sin^2A$ \\

Other trig identities are: \\

$tanx=\frac{sinx}{cosx}$ \\
$sin^2x + cos^2x = 1$ \\
$sin^2x = 1-cos^2x$ \\
$cos^2x = 1+sin^2x$ \\
$\frac{1}{tanx}=cotx$ \\
$cotx=\frac{cosx}{sinx}$ \\
$\frac{1}{cosx}=secx$ \\
$\frac{1}{sinx}=cscx$ \\
$tan^2x+1=sec^2x$ \\
$cot^2x+1=csc^2x$ \\

when proving trig identities, remember to use LS and RS, then make them equal, and then conclude that the proof has been completed (QED). Try to start with the more complicated side, simply to easier expressions like $sinx$ and $cosx$, and be wary of having to factor or use a common denominator. 

\section{Trig Equations}

Linear trig equations are relatively easy to solve. Just be sure to cover all cases (CAST Rule) and keep the answer exact when possible. \\

Quadratic trig equations are easy to solve when you substitute in a "dummy" variable instead of leaving trig expressions. Remember to cover all cases, which can grow quite quickly, especially with abnormal restrictions. \\

When working with restrictions, remember that negative rotations are done in the CAST diagrams, and be wary of inclusion when solving them. \\

Identities are the same as all other trig equations, they simply require more work to get into an easily solvable form. \\

\section{Trig Functions}

It is assumed that you can graph the basic trig functions at this point, if you cannot then hit the books. \\

For translations, the general equation is \\

$f(x)=sin(x-p) +q$, $p = $ phase shift (left/right), $q = vertical translation/shift$. These are relatively simple translations, as they are only shifts in any given direction. \\

Dilatations can be tricky without practice. They are expressed in the form \\

$f(x)=asin(kx)$, $a =$ amplitude, which is just a vertical stretch or compress, and relatively simple. $k =$ periodicity, however be cautious as the period of the function is determined by $T = \frac{2\pi}{k}$, $T =$ period. \\

Determining the equation of a trig function given some information is often just a substitute and solve problem. They may scale in complexity but are still fundamentally the same. \\

Applications of trig functions need to be analyzed. Remember to follow the pre-established formulas, and be cautious when reading the question. Any other questions involving them besides graphing or determining the formula follow previously taught principles. \\

For sketching reciprocal trig functions, first determining the VAs, which occur at $\pi$ when there are no alterations to the formula. They are easiest to graph by graphing the regular functions, then converting to the reciprocal. Note that because of order of operations, the center-most point of each half period is not flipped, but rather at the same spot as the respective peak/trough.

\end{document}
